\section{Inclined Plane Model}

Here, the model is on an inclined plane described by $\zeta\left(\vec{x}\right): \mathbb{R}^2 \rightarrow \mathbb{R}$.

\begin{align*}
b_{t} & =G_{b}b\left(1-b\right)-b+\delta_{b}\nabla^{2}b\\
w_{t} & =Ih-\nu (1-\rho b) w-G_{w}w+\delta_{w}\nabla^{2}w\\
h_{t} & =p-Ih-\nabla\left(-2\delta_{h}h\cdot\nabla\left(h+\zeta\right)\right)\\
 &= p - Ih +2 \delta_h \left( \nabla^2 (h^2) + \nabla h \cdot \nabla \zeta + h\nabla^2 \zeta \right) \\
I & =\alpha\frac{b+qf}{b+q}\\
G_{b} & =\nu w\left(1+\eta b\right)^{2}\\
G_{w} & =\gamma b\left(1+\eta b\right)^{2}
\end{align*}

\subsection{Boundary Conditions}
On the $x$ axis we simply use periodic boundary conditions to model that we are looking at a slice of the slope. However, in the y axis we look at the entire range. At the top $y=L$ we get zero flux (this is the definition of a drainage basin). At the bottom $y=0$, we choose to have zero curvature as to not inhibit the possible flux values in the boundary.

\begin{align*}
    f''\left(x,0\right) = f'(x,L)=0
\end{align*}
where $f$ is every field we have in our model: $b,w,h$.

With this boundary conditions we can find a basis of the form
\[
f_n(y) = A_n \sin\left(\frac{2n+1}{2}\frac{\pi}{L}y\right)+ B_n
\]
for $n=0,1,...$.

\subsection{Uniform States and Zero Biomass}

\subsubsection{Stable States}

Setting $b=0$, and letting $\nabla w,\nabla h$ be anything, we get for a stable
uniform solutions:

\begin{align*}
0 & =\alpha f h-\nu w + \delta_w \nabla^2 w\\
0 & =p - \alpha f h +2 \delta_h \left( \nabla^2 (h^2) + \nabla h \cdot \nabla \zeta + h\nabla^2 \zeta \right) 
\end{align*}
We got an implicit equations for $h,w$, that cannot be solved analytically.

Demanding that the water flux would be the same everywhere we get
\begin{align*}
    h &= \frac{p}{\alpha f}\\
    0 &= p-\nu w + \delta_w \nabla^2 w
\end{align*}
which now means that only $w$ is given implicitly.

\subsubsection{Stability of the Zero Biomass state}
For just one spatial dimension ($y$) we can write the models equations
as:
\begin{align*}
b_{t} & =\nu w\left(1+\eta b\right)^{2}b\left(1-b\right)-b+\delta_{b}\frac{\partial^{2}}{\partial y^{2}}b\\
w_{t} & =\alpha\frac{b+qf}{b+q}h-\nu\left(1-\rho b\right)w-\gamma b\left(1+\eta b\right)^{2}w+\delta_{w}\frac{\partial^{2}}{\partial y^{2}}w\\
h_{t} & =p-\alpha\frac{b+qf}{b+q}h+2\delta_{h}\frac{\partial^{2}}{\partial y^{2}}\left(h^{2}\right)+2\delta_{h}\frac{\partial}{\partial y}h\cdot\frac{\partial}{\partial y}\zeta+2\delta_{h}h\frac{\partial^{2}}{\partial y^{2}}\zeta
\end{align*}

We'll want to find the Jacobian so we'll write the model's equations in the form
\[
\Vec{u}_t = F[\Vec{u}]
\]
and perturb the zero vegetation solution using a small perturbation of the form $\left(\begin{array}{c}
\beta\\
\omega\\
\epsilon
\end{array}\right)e^{-iky}$.
We'll need to have $I$ in an easier form to work with so we'll write:

\begin{align*}
    I(b_0 + \delta b) &\approx I(b_0) + I'(b_0) (b_0 - \delta b) \\
    &= \alpha \frac{b_0 + q f}{b_0+q} + \alpha q \frac{1-f}{(b_0+q)^2}(b_0-\delta b)
\end{align*}

Pluging this form into the model equations and taking all the terms of order $\beta~\omega~\epsilon$ we can get the following Jacobian:
\[
J_0 = \left[\begin{matrix}- \delta_{b} k^{2} + p - 1 & 0 & 0\\- \frac{\gamma p}{\nu} + \rho p + \frac{p}{q} - \frac{p}{f q} & - \delta_{w} k^{2} - \nu & \alpha f\\- \frac{p}{q} + \frac{p}{f q} & 0 & - \alpha f - 2 i \delta_{h} k \frac{d}{d y} \zeta{\left(y \right)} + 2 \delta_{h} \frac{d^{2}}{d y^{2}} \zeta{\left(y \right)} - \frac{2 \delta_{h} k^{2} p}{\alpha f}\end{matrix}\right]
\]

The eigenvalues are:

\begin{align*}
\sigma_{1} & =p-1-\delta_{b}k^{2}\\
\sigma_{2} & =-\nu-\delta_{w}k^{2}\\
\sigma_{3} & =-\alpha f-2\delta_{h}\frac{p}{\alpha f}k^{2}+2\delta_h\frac{d^2\zeta}{dy^2}-2i\delta_{h}mk
\end{align*}

$\sigma_2<0$ for all $k\geq0$ so this mode is always stable. The first mode, $\sigma_1$ is stable only for $p<1$. For the last mode, $\sigma_3$, we'll look only at the real part:
\begin{align*}
    -\alpha f - 2 \delta_h \frac{p}{\alpha f}k^2 + 2\delta_h\frac{d^2\zeta}{dy^2} < 0
\end{align*}https://www.overleaf.com/project/630b4d694fb1bb63d7a60b6f
The left hand side is maximal when $k=0$, which means that
\begin{align*}
    -\alpha f + 2\delta_h\frac{d^2\zeta}{dy^2} < 0 \\
    \frac{d^2 \zeta}{dy^2} < \kappa_c = \frac{\alpha f}{2\delta_h}
\end{align*}

The value $\kappa_c$ is the critical value; when the curvature is bigger than it, the zero state is never stable (\textbf{What is the physical meaning of this?}).

In the simulations we do see at the start a uniform change in the
system \textbf{(need a good graph here of $\left|b\right|$ and $Var\left(b\right)$
over time)} until some point further in time where modes that are
bigger than zero starts to grow, transforming the system into a stationary
non-uniform state \textbf{(needs to understand that transition)}.

\subsection{Uniform Nonzero States}
It is not possible to find explicit form of the general stationary uniform solution.
