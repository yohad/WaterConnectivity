\documentclass{article}
\usepackage{graphicx, float} % Required for inserting images
\usepackage{enumitem,amssymb,amsmath}
\usepackage{fullpage}
\usepackage{multicol}
\usepackage{hyperref}
\usepackage{caption, subcaption}

\usepackage[style=authoryear,sorting=nyt,maxnames=1,minnames=1,backend=biber]{biblatex}
\addbibresource{biblatex/WaterConnectivity.bib}
\addbibresource{biblatex/code.bib}

% Setup of hyperlinks incompilation
\usepackage{hyperref}
\hypersetup{colorlinks=true,allcolors=black}
\usepackage{hypcap}

\numberwithin{equation}{section}


\title{Relation Between Biomass Patterns and Water Flux in an Arid Drainage Basin System}

\author{Yotam Ohad \and Prof. Ehud Meron}

\begin{document}

\maketitle

\begin{abstract}
    In this work we study the relations between the water flux flowing out of a drainage basin to the patterns formed by its vegetation. We find that the results agrees with previous work \parencite[]{meron_vegetation_2004}, about the relation between the vegetation patterns to the water availability gradient, and offer a way to relate the vegetation patterns to the water flux.
\end{abstract}

\section{Introduction}
In the study of dryland ecosystems, it was found that the basic block one needs to consider is the drainage basin \parencite{yair_case_1982}. The drainage basin has two main components: the slope and the riverbed downhill, which is commonly a dry riverbed. The biomass ratio between the slope and the riverbed has not yet been the subject of studies addressing the effects of climate change and severe droughts on vegetation (corroborated by Prof. Moshe Shachak). This ratio depends to a large extent on the patchy nature of the vegetation that develops on the slope, which tends to self-organizes in nearly periodic patterns for nearly homogeneous slopes (\cite{lefever_origin_1997}, \cite{valentin_soil_1999}, \cite{klausmeier_regular_1999}, \cite{von_hardenberg_diversity_2001}).

A commonly observed vegetation pattern on slopped terrains is banded vegetation consisting of alternating stripes of vegetation and bare soil, oriented perpendicular to the slope direction. In this configuration each stripe benefits not only from direct rainfall but also from the interception of runoff water generated in the bare-soil stripe uphill, where the infiltration rate of surface water into the soil is lower than the rate in the vegetation stripe \parencite{meron_vegetation_2019}. However, at sufficiently lower precipitation rates vegetation stripes break into arc-shaped vegetation spots and concomitantly bare-soil areas merge to form longer pathways for water flow downhill. These pathways, in turn, increase the seepage of water and nutrients from the slope to the riverbed and change the ratio of vegetation biomass between the two components. Longer pathways of water and nutrient flow are often described as increased connectivity \parencite{okin_connectivity_2015}

\section{The Model}
\subsection{Model's Equations}
We will study the problem of biomass patterns in drainage basin using the model proposed by \cite{gilad_phys_2004}, and \cite{gilad_mathematical_2007}. We choose this model as, unlike simpler models, it makes a distinction between soil water and surface water. Therefore distinguishing the surface water from the rest of the system, enabling us to examine it and water flux which derives from it.

The model can be written in a non-dimensional form as such:
\begin{align}
    \partial_t b & = G_b(x,y) b(1-b) - b + \delta_b\nabla^2 b              \\
    \partial_t w & = Ih - \nu(1-\rho b)w - G_w(x,y)w + \delta_w \nabla^2 w \\
    \partial_t h & = p - Ih - \nabla \cdot \vec{J}
\end{align}
where $b(x,y),w(x,y),h(x,y)$ are the non-dimensional functions from $\mathbb{R}^2$ to $\mathbb{R}$ describing the biomass, soil water, and surface water. $\vec{J}$ is the (non-dimensional) water flux and is given by:
\begin{align}
    \label{eq:flux_def}
    \vec{J} =  -2\delta_h h\nabla\left(h+\zeta(x,y)\right)
\end{align}
using $\zeta$ as the topography function of the slope. Assuming uniform slope $\zeta(x,y)=my\label{eq:terrain}$, the water flux can be written as:
\begin{equation}
    \nabla \cdot \vec{J} = -\delta_h \nabla^2(h^2) - 2\delta_h m \partial_y h
\end{equation}
$I$ is the infiltration rate and is defined using
\begin{equation}
    I = \alpha \frac{b+qf}{b+q}
\end{equation}
This form captures the dependence of the infiltration rate on the biomass. $f\in[0,1]$ and $q\in(0,\infty)$, so for large values of $f$ (close to 1), the infiltration rate doesn't depends much on the biomass, while for small values of $f$ (close to 0) the dependence is extremely large - $I\approx \alpha\frac{b}{b+q}$.

The Functions $G_b,G_w$ are defined using the kernel function
\begin{equation}
    g(\mathbf{x}, \mathbf{x'},t) = \frac{1}{2\pi} \exp\left[-\frac{\left|\mathbf{x}-\mathbf{x'}\right|^2}{2(1+\eta b(\mathbf{x},t))^2}\right]
\end{equation} as \begin{align}
    G_b(\mathbf{x},t) & = \nu \int_\Omega g(\mathbf{x},\mathbf{x'},t)w(\mathbf{x'},t)d\mathbf{x'}    \\
    G_w(\mathbf{x},t) & = \gamma \int_\Omega g(\mathbf{x'},\mathbf{x},t)b(\mathbf{x'},t)d\mathbf{x'}
\end{align}
We add the assumption the the root zone is narrowly confined. In the dimensional form this means that $S_0\rightarrow 0$ \parencite[see][eq (3)]{gilad_phys_2004}. Using this assumption we get:
\begin{align}
    G_b & = \nu w (1+\eta b)^2   \\
    G_w & = \gamma b(1+\eta b)^2
\end{align}
Our model equations are then:
\begin{align}
    \partial_t b & = \nu bw(1-b)(1+\eta b)^2 - b + \delta_b\nabla^2 b                   \\
    \partial_t w & = I h - \nu(1-\rho b)w - \gamma bw(1+\eta b)^2 + \delta_w \nabla^2 w \\
    \partial_t h & = p - I h - \nabla \cdot \vec{J}
\end{align}

\subsection{Scales of the slope}
From equation \ref{eq:flux_def}, assuming uniform slope as in equation \ref{eq:terrain}, we can write the time derivative of $h$ in the following manner:
\begin{equation}
    \partial_t h = \frac{1}{1 - 2\delta_h m h} \left(p - I h + 2\delta_h h\nabla h\right)
\end{equation}
From this, we can see that if $0 < 2\delta_h m h \ll 1$ then the system dynamics do not depend on the terrain's slope. This inequality will be taken into account when choosing the range of slope values to simulate over.

\subsection{Uniform Bare Steady State Solution}
To be able to verify that our simulation indeed represents the mathematical model correctly, we derive in this section the linear stability of the uniform bare steady state solution. This linear stability can then be checked in the simulation.
\subsubsection{Solutions of the Model}
The uniform bare steady state solution of the model is a solution that has no vegetation at all, and homogeneous surface and soil water. In equation form we can write this as:
\begin{align}
    b(x,y)                      & =0  \\
    \partial_t w = \partial_t h & = 0 \\
    \nabla w = \nabla h         & = 0
\end{align}
From the first condition, we can see that both $\partial_tb=0$ and $\nabla b=0$ holds. We are left with the following system of equations:
\begin{align}
    0 & = \alpha h - \nu w \\
    0 & = p-\alpha f h
\end{align}
which admits the following solutions:
\begin{align}
    w_0 & = \frac{p}{\nu}      \\
    h_0 & = \frac{p}{\alpha f}
\end{align}
\subsubsection{Linear Stability Analysis}
We start by writing the right hand side of the model equations in operator form and equating it to zero to find the steady state.
\begin{align}\label{eqn:functional_eq}
    \vec{0}=\vec{F}[b,w,h]
\end{align}
To find the linear stability of the uniform bare steady state solution, we'll calculate the linear approximation of $\vec{F}$  when perturbing the solution we found using $\begin{bmatrix}
        \varepsilon_b(t) \\ \varepsilon_w(t) \\ \varepsilon_h(t)
    \end{bmatrix}e^{i\vec{k}\cdot\vec{r}}$, an infinitesimal perturbations with $0<|\varepsilon_s|\ll1$ for $s\in \{b,w,h\}$.  $\vec{r}=(x,y)$ is the position vector in the region, and $\vec{k}=(k_x,k_y)$ is the vector wave number of the perturbation ($|\vec{k}|\geq 0$).

Plugging this form into \ref{eqn:functional_eq}, we get the following relations:
\begin{align}\label{eps_eqs}
    \partial_t \varepsilon_b & = (-\delta_b k^2+p - 1)\varepsilon_b + o(\varepsilon^2)                                                     \\
    \partial_t \varepsilon_w & = \frac{p}{q}(1-f)\varepsilon_b + (-\delta_wk^2-\nu)\varepsilon_w +\alpha f\varepsilon_h + o(\varepsilon^2) \\
    \partial_t \varepsilon_h & = (-\alpha -2\delta_h \frac{p}{\alpha} k^2 +2i\delta_hk_ym) \varepsilon_h + o(\varepsilon^2)
\end{align}
where $o(\varepsilon^2)$ are nonlinear terms in $\varepsilon_{b,w,h}$. We assume the initial perturbations are infinitesimal, we can ignore nonlinear terms in \ref{eps_eqs}. We are left with the following linear ODE:
\begin{equation}
    \partial_t \begin{bmatrix}
        \varepsilon_b \\ \varepsilon_w \\ \varepsilon_h
    \end{bmatrix} = \begin{bmatrix}
        -\delta_b k^2+p - 1 & 0                & 0                                                       \\
        \frac{p}{q}(1-f)    & -\delta_wk^2-\nu & \alpha f                                                \\
        0                   & 0                & -\alpha -2\delta_h \frac{p}{\alpha} k^2 +2i\delta_hk_ym
    \end{bmatrix} \begin{bmatrix}
        \varepsilon_b \\ \varepsilon_w \\ \varepsilon_h
    \end{bmatrix}
\end{equation}
Solutions of these equation are linear combinations of the eigenvectors of the matrix. Each eigenvector grows if its eigenvalue is positive, and decays for negative eigenvalues. Linearly stable states would have all the eigenvalues as  negative. In this problem, the eigenvalues can be easily found to be:
\begin{align}
    \sigma_1 & = -\delta_b k^2+p - 1 \label{eq:sigma1}                   \\
    \sigma_2 & = -\delta_w k^2 -\nu                                      \\
    \sigma_3 & = -\alpha -2\delta_h \frac{p}{\alpha} k^2 +2i\delta_hk_ym
\end{align}
We see that the real parts of $\sigma_2,\sigma_3$ are negative for all $k^2 = k_x^2+k_y^2>0$. For $p<1$ we get $\sigma_1<0$  while for $p>1$  there exists a range of values of $k^2$  for which $\sigma_1>0$ and the steady state loses stability. The point at which this happens is denoted $p_c$ (critical precipitation) and can be found by solving
\begin{align}
    \sigma_1             & = 0  \\
    \frac{d\sigma_1}{dk} & =  0
\end{align}
The second equation gives $k=0$ which then leads to $p=1$. Then $p_c=1$ (a local maximum of $\sigma_1)$. For $p<1$ then, the perturbations decrease while for $p>1$ the perturbations grow, changing the state of the system into a non-zero biomass state. We could the $p$ parameter in our numerical code, to check for the validity of our simulations.
We note here that the slope, in the case of linear terrain, only appears in the complex part of the eigenvalues and thus doesn't effect the linear stability. Its meaning comes into play only when the system is non-homogeneous. In this case, it relates to the speed of which spatial perturbation travel up the slope.
\section{Simulating the model}

For the numerical simulations of the model, we used the python package \texttt{dedalus3} \parencite[]{dedalus2020}. The code used in this paper, and scripts to generate the data used, can be found at \cite{Ohad_Relation_Between_Biomass_2023}.

\subsection{Boundary Conditions}
We now move to discuss how to simulate the model. For simplicity we choose a square region of $N\times N$ points, relating to an area of  $L\times L$  we wish to simulate. For the $x$ coordinates, we have periodic boundary conditions. For the $y$ axis, we wish to not force the total flux coming out of the region. This leads as to consider Neumann boundary condition at the bottom edge for the flux. Using $h(x,y=0)=h$ we can write $J_y=-2\delta_h h \frac{\partial}{\partial y}(h+\zeta)$ as the $y$ compononent of the flux. We can then write the boundary condition as:
\begin{equation}
    \frac{\partial J_y}{\partial y}(y=0) = 0
\end{equation}
or in terms of $h$:
\begin{equation}
    \left(\frac{\partial h}{\partial y}\right)^2 + h \frac{\partial^2 h}{\partial y^2} + m \frac{\partial h}{\partial y} = 0
\end{equation}

Using $H=\frac{\partial h}{\partial y}$ we can get a Neumann boundary condition for $h$, i.e., $\frac{\partial h}{\partial y}=A\in \mathbb{R}$.
\begin{align}
    \left(H + m + h\frac{\partial}{\partial y}\right)H         & = 0             \\
    \Rightarrow \frac{1}{H(H+m)} \frac{\partial H}{\partial y} & = - \frac{1}{h}
\end{align}
Integrating both sides over the range of $y$ values we get:
\begin{equation}
    \frac{1}{m}\ln{\frac{H}{H+m}} = -\int_0^{L_y} \frac{dy}{h}
\end{equation}
We can isolate $H$
\begin{equation}
    \frac{\partial h}{\partial y} (y=0) = H = m \frac{e^{-m \int_0^{L_y} \frac{dy}{h}}}{1 - e^{-m \int_0^{L_y} \frac{dy}{h}}}
\end{equation}

For other fields and edges in the $y$ coordinate, we chose a zero Neumann boundary condition - $\frac{\partial f}{\partial y}=0$.
Note that this additional condition makes the flux at the top edge equals to $mh$. This means our top edge isn't the separation line between drainage basins.

\subsection{Results}
The values of the constants of the non-dimensional model were taken from \parencite[]{gilad_mathematical_2007}.
The values used are shown in Table  \ref{table:constants_tables}.
\begin{table}[!ht]
    \centering
    \begin{tabular}{||cc||}
        \hline
        constant   & value           \\
        \hline\hline
        $m$        & $[0, 10]$       \\
        $p$        & $[0.9, 1.5]$    \\
        $\nu$      & $\frac{10}{3}$  \\
        $\eta$     & $3.5$           \\
        $\rho$     & $0.95$          \\
        $\gamma$   & $\frac{50}{3}$  \\
        $\alpha$   & $0.1$           \\
        $q$        & $0.05$          \\
        $f$        & $0.01$          \\
        $\delta_b$ & $\frac{1}{30}$  \\
        $\delta_w$ & $\frac{10}{3}$  \\
        $\delta_h$ & $\frac{1}{300}$ \\
        \hline
    \end{tabular}
    \caption{Values of simulation's constants.}
    \label{table:constants_tables}
\end{table}

First, we checked that the numerical simulation indeed reproduces the linear stability analysis. The simulation started from the homogeneous non-bare state, and was left to run for 500 years. The precipitation values where chosen to be around $p_c$ (the critical precipitation). Figure \ref{fig:critical_point} shows the results of the simulation. We see that for $p<p_c$ the system decays to the zero state, while for $p>p_c$ the system moves to a nonzero state. This is in agreement with the linear stability analysis. Below $p_c$ the perturbations decay, while above $p_c$ they grow.
\begin{figure}[!ht]
    \centering
    \includegraphics[scale=0.5]{plots/critical point.png}
    \caption{Simulations around $p_c$ (different scales)}
    \label{fig:critical_point}
\end{figure}

We now move to study the water flux in the system in the case of $p>p_c$. We started each simulation from the nonzero uniform state, where the precipitation was chosen to be lower than the stationary state precipitation.
We let the simulation run for 500 years, and plotted the flux going out of the region.
We simulated different slopes (figure \ref{fig:flux_out_slopes}) and different precipitation values (figure \ref{fig:flux_out_precipitation}).

\begin{figure}[!ht]
    \centering
    \caption{Flux out of the region for various values of $m$.}
    \label{fig:flux_out_slopes}

    \begin{subfigure}[]{0.5\textwidth}
        \centering
        \includegraphics[scale=0.3]{plots/p1_1_m1_0.png}
        \phantomcaption
        \label{fig:m1_0}
    \end{subfigure}

    \begin{subfigure}[]{0.5\textwidth}
        \centering
        \includegraphics[scale=0.5]{plots/p1_1_m50_0.png}
        \phantomcaption
        \label{fig:m50_0}
    \end{subfigure}

    \begin{subfigure}[]{0.5\textwidth}
        \centering
        \includegraphics[scale=0.5]{plots/p1_1_m100_0.png}
        \phantomcaption
        \label{fig:m100_0}
    \end{subfigure}
\end{figure}

\begin{figure}[!ht]
    \centering
    \caption{Flux out of the region for various values of $p$.}
    \label{fig:flux_out_precipitation}

    \begin{subfigure}[]{0.5\textwidth}
        \centering
        \includegraphics[scale=0.5]{plots/p1_15_m1_0.png}
        \label{fig:p1_15}
        \phantomcaption
    \end{subfigure}

    \begin{subfigure}[]{0.5\textwidth}
        \centering
        \includegraphics[scale=0.5]{plots/p1_17_m1_0.png}
        \label{fig:p1_17}
        \phantomcaption
    \end{subfigure}

    \begin{subfigure}[]{0.5\textwidth}
        \centering
        \includegraphics[scale=0.5]{plots/p1_19_m1_0.png}
        \label{fig:p1_19}
        \phantomcaption
    \end{subfigure}

\end{figure}

\section{Discussion}

Figure \ref{fig:critical_point} Shows that $p_c$ is the critical precipitation in the numerical simulation.
Furthermore, The first mode to grow in the simulations is the nonzero uniform state, which is related to the dispersion curve \ref{eq:sigma1}, where the first mode to grow is the uniform mode, $k=0$.
Those phenomena where anticipated by the model, and assure us that our numerical simulation is correct.

In figure \ref{fig:flux_out_precipitation} we see the expected result, that as we increase the precipitation, the final pattern in the simulation is higher in the rainfall gradient. Furthermore, we can see the higher the precipitation, the lower the outward flux. This suggest that higher patterns in the rainfall gradient are more efficient in absorbing water.
We explain the increased outward water flux to a higher degree of water connectivity in the hexagonal spots, and in a lesser extent stipes and holes. That is to say, water can move farther without being absorbed by the soil or vegetation.

Increasing the slope in the simulations (figure \ref{fig:flux_out_slopes}) yields different results than increasing the precipitation. We see that the higher the slope, the less stable the new outwardflux is, after the transition from the uniform nonzero state. While the overall dynamics of the system, uniform nonzero state to hexagonal pattern, remains the same. This suggest that the slope doesn't effect the system dynamics, but does effect the outward flux.

In areas that goes through droughts or desertification, precipitation becomes lower. We see from our simulations that this can be a cause to the higher rate of floods occuring in such area. The lower precipitation causes the system to move to a lower pattern in the rainfall gradient, which in turn causes a higher connectivity in the surface water. This higher connectivity apears as increased floods.

\section{Bibliography}
\printbibliography

\end{document}
